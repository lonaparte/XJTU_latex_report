\documentclass[a4paper,12pt]{report}
\usepackage[UTF8]{ctex}
\usepackage{lipsum}
\usepackage{listings} 
\usepackage{bigstrut}
\usepackage{graphicx}
\usepackage[left=2.5cm,right=2.5cm,top=2cm,bottom=3.5cm]{geometry}
\usepackage{wallpaper} 
\usepackage{fancyhdr}
\usepackage{titlesec}
\usepackage{titletoc}
\usepackage[titletoc]{appendix}
\usepackage{times}
\usepackage{setspace}
\usepackage{wrapfig}
\usepackage{array}
\usepackage{cite}
\usepackage[hidelinks]{hyperref}
% \usepackage[framed,numbered,autolinebreaks,useliterate]{mcode}

\newcommand{\subject}[1]{\renewcommand{\subject}{#1}}
\newcommand{\titre}[1]{\renewcommand{\titre}{#1}}
\newcommand{\name}[1]{\renewcommand{\name}{#1}}
\newcommand{\numero}[1]{\renewcommand{\numero}{#1}}
\newcommand{\classnum}[1]{\renewcommand{\classnum}{#1}}
\newcommand{\semester}[1]{\renewcommand{\semester}{#1}}

\titlecontents{chapter}[0em]{\songti\zihao{-4}}{\thecontentslabel\ }{}
{\hspace{.5em}\titlerule*[4pt]{$\cdot$}\contentspage}
\titlecontents{section}[2em]{\vspace{0.1\baselineskip}\songti\zihao{-4}}{\thecontentslabel\ }{}
{\hspace{.5em}\titlerule*[4pt]{$\cdot$}\contentspage}
\titlecontents{subsection}[4em]{\vspace{0.1\baselineskip}\songti\zihao{-4}}{\thecontentslabel\ }{}
{\hspace{.5em}\titlerule*[4pt]{$\cdot$}\contentspage}

\newcommand{\fig}[4]{
\begin{figure}[ht]
\centering
\includegraphics[width=#2\textwidth]{images/#1}
\caption{#3}
\label {fig: #4}
\end{figure}
}

\newcommand{\eq}[2]{
\begin{equation}
#1
\label {eq: #2}
\end{equation}
}

\newcommand{\figfig}[8]{
\begin{figure}[ht]
\begin{minipage}{#2\textwidth}
\centering
\includegraphics[width=1\textwidth]{images/#1}
\caption{#3}
\label {fig: #4}
\end{minipage}
\begin{minipage}{#6\textwidth}
\centering
\includegraphics[width=1\textwidth]{images/#5}
\caption{#7}
\label {fig: #8}
\end{minipage}
\end{figure}
}

\begin{document}

% -----------------------------------------------------------------------
% 文档信息
% -----------------------------------------------------------------------
\subject{微机原理}
\titre{实验报告}
\name{无名氏}
\numero{6666666666}
\classnum{电气45班}
\semester{2018年秋季学期}
% -----------------------------------------------------------------------

% -----------------------------------------------------------------------
% 自定义章节格式
% -----------------------------------------------------------------------
\titleformat{\chapter}{\centering\zihao{-1}\heiti}{实验\chinese{chapter}}{1em}{}
\titlespacing{\chapter}{0pt}{*0}{*6}
% -----------------------------------------------------------------------

% -----------------------------------------------------------------------
% 自定义摘要标题
% -----------------------------------------------------------------------
\renewcommand{\abstractname}{\zihao{-3} 摘\quad 要}
% -----------------------------------------------------------------------

% -----------------------------------------------------------------------
% 自定义参考文献格式
% -----------------------------------------------------------------------
% \renewcommand{\bibname}{\zihao{2}{\hspace{\fill}参\hspace{0.5em}考\hspace{0.5em}文\hspace{0.5em}献\hspace{\fill}}}
\makeatletter
\def\@cite#1#2{\textsuperscript{[{#1\if@tempswa , #2\fi}]}}
\makeatother
% -----------------------------------------------------------------------

\fancypagestyle{plain}{
	\pagestyle{fancy}
}
\pagestyle{fancy}
\fancyheadoffset{1cm}
\setlength{\headheight}{2cm}
% -----------------------------------------------------------------------
% 页眉和页脚设置
% -----------------------------------------------------------------------
% \renewcommand{\chaptername}{实验\chinese{chapter}}
% \renewcommand{\chaptermark}[1]{\markboth{\chaptername\ #1}{}}
\renewcommand{\chaptermark}[1]{\markboth{#1}{}}
\lhead{\includegraphics[scale=0.1]{logos/logo_xjtu.png}} %Affichage de l'image au top de la page
\rhead{\protect\nouppercase\leftmark}
\cfoot{\thepage}
\chead{\subject}
\lfoot{\titre}
\rfoot{\name \ \ \numero}
% -----------------------------------------------------------------------

% -----------------------------------------------------------------------
% 代码格式设置
% -----------------------------------------------------------------------
\lstset{numbers=left, 
	basicstyle=\small\sffamily,
	numbers=left,
 	numberstyle=\tiny,
	frame=none,
	tabsize=4,
	columns=fixed,
	showstringspaces=false,
	showtabs=false,
	keepspaces,
	commentstyle=\color{red},
	keywordstyle=\color{blue},
    breaklines=true,
    backgroundcolor=\color{lightgray!40!white},
    basicstyle=\ttfamily,
    commentstyle=\ttfamily\color{green!40!black}
} 
% -----------------------------------------------------------------------

\begin{titlepage}

\ThisLRCornerWallPaper{0.9}{logos/background.jpg}
\centering %Centraliser le contenu
\includegraphics[width=0.8\textwidth]{logos/logo_xjtu.png}\par\vspace{1cm} 
\vspace{1.5cm}
\rule{\linewidth}{0.4 mm} \\[0 cm]
{\Huge\bfseries \subject \par} %sous-titre
\vspace{0.2cm}
\rule{\linewidth}{0.4 mm} \\[0.4 cm]
{\LARGE\bfseries \titre \par} \
\\
{\large \semester \par} \
\\
\vspace{3cm}
\large{
\begin{tabular}{ll}
姓名&\name \\
学号&\numero \\
班级&\classnum \\
\end{tabular}}
\vspace{2cm}
    
\vfill
{\large \today\par} %Affichage de la date

\end{titlepage}

% -----------------------------------------------------------------------
% 摘要页
% -----------------------------------------------------------------------
\begin{abstract}
\begin{spacing}{1.5}
    {\zihao{-4}
    西安交通大学\\[0.5cm]
    \textbf{关键字}:\quad 西安交通大学
    }
\end{spacing}
\end{abstract}
% -----------------------------------------------------------------------

\tableofcontents
\newpage

\chapter{实验一}

\section{介绍}

% -----------------------------------------------------------------------
% 参考文献
% -----------------------------------------------------------------------
\addcontentsline{toc}{chapter}{参考文献}

\begin{thebibliography}{99}
\songti \zihao{-4} 	
\bibitem{Leslie.{1994}}
Leslie Lamport. LATEX: A Document Preparation System.AddisonWesley, Reading, Massachusetts, second edition, 1994, ISBN 0-201-52983-1.

\bibitem{Donald.{1984}}
Donald E. Knuth. The TEXbook, Volume A of Computers and Typesetting,Addison Wesley, Reading, Massachusetts, second edition, 1984,ISBN 0-201-13448-9.
\end{thebibliography}
% -----------------------------------------------------------------------

% -----------------------------------------------------------------------
%  附录设置
% -----------------------------------------------------------------------
\titleformat{\chapter}{\heiti\Large}{附录~\Alph{chapter}}{11pt}{\Large}
\titlespacing{\chapter}{0pt}{*-4}{*4}

\lstset{breaklines}                %自动将长的代码行换行排版
\lstset{extendedchars=false}
\lstset{language=Matlab}
\renewcommand{\thechapter}{附录\Alph{chapter}.} 
\appendix
\begin{appendix}
	
	
\chapter{数据表}
\zihao{-4}\songti
\begin{spacing}{1.5}
	hello world!
\end{spacing}


\chapter{程序代码}
\zihao{-4}\songti
\begin{spacing}{1.5}
下面是一个MATLAB程序的事例,使用了Package mcode,它能较好还原MATLAB本身的编写风格。
\begin{lstlisting}[language={matlab}]
%The program normalizes the measurement data and compares it to the standard cosine function
data=xlsread('data_sun',1,'B3:E39');
min=[(data(1,1)+data(37,1))/2,(data(1,2)+data(37,2))/2,...
(data(1,3)+data(37,3))/2,(data(1,4)+data(37,4))/2];
max=[data(19,1),data(19,2),data(19,3),data(19,4)];
Min=repmat(min,37,1);
Max=repmat(max,37,1);
data=(data-Min)./(Max-Min);
x=-pi/2:pi/36:pi/2;
y=cos(x);
%----------------------figure-------------------------%
figure(1);
subplot(2,2,1);
plot(x,data(:,1),'ro-');
hold on;
plot(x,y,'b-');
title('R=1.2\Omega');
axis([-2,2,0,1]);
grid on;
subplot(2,2,2);
plot(x,data(:,2),'ro-');
hold on;
plot(x,y,'b-');
title('R=1.6\Omega');
axis([-2,2,0,1]);
grid on;
subplot(2,2,3);
plot(x,data(:,2),'ro-');
hold on;
plot(x,y,'b-');
title('R=2.0\Omega');
axis([-2,2,0,1]);
grid on;
subplot(2,2,4);
plot(x,data(:,4),'ro-');
hold on;
plot(x,y,'b-');
title('R=2.4\Omega');
grid on;
axis([-2,2,0,1]);
\end{lstlisting}
\end{spacing}
\end{appendix}
% -----------------------------------------------------------------------

\end{document}